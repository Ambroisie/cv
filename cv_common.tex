%-------------------------------------------------------------------------------
% PERSONAL INFORMATION
%-------------------------------------------------------------------------------

\profilepic{} % Profile picture

\cvname{Bruno BELANYI} % Your name
\cvjobtitle{Computer Science Student} % Job title/career

\cvdate{\enfr{04 November 1999}{04 Novembre 1999}} % Date of birth
\cvaddress{Kremlin-Bicêtre, France} % Current address
\cvmail{bruno@belanyi.fr} % Mail address
\cvnumberphone{+33 7 81 59 44 86} % Phone number
\cvsite{https://belanyi.fr} % Personal website

%-------------------------------------------------------------------------------

\begin{document}

%-------------------------------------------------------------------------------
% ABOUT ME
%-------------------------------------------------------------------------------

\aboutme{%
  \enfr{%
    Just graduated from EPITA, I am looking for a full-time position.
  }{%
    Nouveau diplomé d'EPITA, je suis à la recherche d'un travail en temps plein.
  }
}

%-------------------------------------------------------------------------------
% Languages
%-------------------------------------------------------------------------------

\languages{%
  {\enfr{French}{Français} / \enfr{Native}{Langue Natale}},%
  {\enfr{English}{Anglais} / \enfr{Fluent}{Couramment} (C2)}%
} % Simple list

%-------------------------------------------------------------------------------
% SKILLS
%-------------------------------------------------------------------------------

% Skill section
\programminglanguages{%
  {C++},%
  {Rust},%
  {C},%
  {Python},%
  {Java},%
  {SQL},%
  {Bash}%
}

\programmingskills{%
  {\enfr {Algorithms}{Algorithmes}},%
  {\enfr{Data Stuctures}{Structures de Données}},%
  {\enfr{Parallel Programming}{Programmation Parallèle}}%
}

\techskills{%
  {Linux \& UNIX},%
  {Git},%
  {Docker},%
  {CI/CD},%
  {Nix \& NixOS},%
  {Jupyter Notebook},%
  {Microsoft Office}%
}

\softskills{%
  {\enfr{Independent}{Indépendant}},%
  {\enfr{Team work}{Travail d'équipe}}%
}

\makeprofile % Print the sidebar

%-------------------------------------------------------------------------------
% INTERESTS
%-------------------------------------------------------------------------------

\section{\enfr{Interests}{Intérêts}}

\enfr{%
  I am interested in computer science and engineering in a broad sense. Always
  checking out new technologies and programming languages that pique my
  interest. I am interested in all topics relating to high-performance
  computing and optimisation.
}{%
  Je m'intéresse à l'informatiques et l'ingénieries au sens large. Je participe
  à la veille technologique, et me tiens au courant de nouveaux langages de
  programmations qui attisent ma curiosité. Je suis intéressé à tous sujets
  traitant de calcul de haute performance et d'optimisation.
}

%-------------------------------------------------------------------------------
% EDUCATION
%-------------------------------------------------------------------------------

\section{Education}

\begin{twenty} % Environment for a list with descriptions
  %\twentyitem{<dates>}{<title>}{<location>}{<description>}
  %\twentyitemshort{<dates>}{<title/description>}

  \twentyitem{2018-2021}
    {\enfr{Computer Science}{Ingénieur Informatique}}
    {EPITA, Kremlin-Bicêtre, France}
    {%
      \enfr%
        {Majoring in Image Processing.}%
        {Majeure de Traitement et Synthèse d'Image.}%
    }

  \twentyitem{2016-2018}
    {\enfr{Preparatory School}{École Préparatoire}}
    {CPGE Corneille, Rouen, France}
    {MPSI \& MP$\star$ --- \enfr{%
        Maths and Physics Engineering Preparatory Classes,
        specialized in Computer Science.
      }{%
        Maths et Physiques, spécialisé en Informatique
      }%
    }

  \twentyitem{2013-2016}
    {\enfr{High School}{Lycée}}
    {Institution Jean-Paul II, Rouen, France}
    {\enfr{Bacalauréat S with honors.}{Bacalauréat S mention Bien.}}

\end{twenty}

%-------------------------------------------------------------------------------
% EXPERIENCE
%-------------------------------------------------------------------------------

\section{Experience}

\begin{twenty}

  \twentyitem{\mar'21-\sep'21}
    {\enfr{Software Engineer}{Ingénieur Informatique}}
    {IMC BV, Amsterdam, \enfr{Netherlands}{Pays-Bas}}
    {%
      \enfr{%
        Part of the Global Execution team.
      }{%
        Membre de l'équipe d'Exécution Globale.
      }
    }

  \twentyitem{\jul'20-\jan'21}
    {\enfr{Teaching Assistant}{Assistant} (ACU)}
    {EPITA, Kremlin-Bicêtre, France}
    {%
      \enfr{%
        Tutoring 3rd year students in C, UNIX, \& software development through
        workshops, oral presentations, and programming projects.
      }{%
        Enseignemt du C, UNIX, et des fondamentaux de la programmations aux
        élèves de 3ème année à l'aide de travaux pratiques, présentations
        orales, et projets.
      }
    }

  \twentyitem{\jan-\jul'20}
    {\enfr{Teaching Assistant}{Assistant} (YAKA)}
    {EPITA, Kremlin-Bicêtre, France}
    {%
      \enfr{%
        Tutoring 3rd year students in C++, Java, \& SQL through oral
        presentations and programming projects.
      }{%
        Enseignemt du C++, Java, \& SQL aux élèves de 3ème année à l'aide de
        travaux pratiques, présentations orales, et projets.
      }
    }

  \twentyitem{\sep-\dec'19}
    {%
      \enfr%
        {Natural Language Processing}
        {Traitement Automatique du Langage Naturel}
    }
    {Algolia, Paris, France}
    {%
      \enfr{%
        Building a Semantic Decompounding Library for Python \& C++
      }{%
        Écriture d'une bibliothèque de décomposition sémantique des mots
        en Python \& C++.
      }
    }

\end{twenty}

%-------------------------------------------------------------------------------
% PROJECTS
%-------------------------------------------------------------------------------


\section{\enfr{School Projects}{Projets Scolaires}}

\begin{twenty}

  \twentyitem{\enfr{2nd Year Ing.}{2\textsuperscript{ème} An.\ d'Ing.}}
    {Pathtracer}
    {Rust - 1 month}
    {%
      \enfr{%
        A physically based renderer of scenes described in a custom YAML format.
      }{%
        Un moteur de rendu graphique réaliste, à partir d'un format YAML.
      }
    }

  \twentyitem{\enfr{1st Year Ing.}{1\textsuperscript{ère} An.\ d'Ing.}}
    {Tiger Compiler}
    {C++ - 3 months}
    {%
      \enfr{%
        A full featured compiler in modern C++, following Andrew Appel's
        \textit{Modern Compiler Implementation in ML}
      }{%
        Un compilateur complet écrit en C++ moderne, en suivant le livre
        d'Andrew Appel \textit{Modern Compiler Implementation in ML}
      }
    }

  \twentyitem{\enfr{1st Year Ing.}{1\textsuperscript{ère} An.\ d'Ing.}}
    {Bistromathique}
    {C++ - \enfr{36 hour rush}{Rush de 36 heures}}
    {%
      \enfr{%
        A templated big-integer library and calculator, completed individually.
      }{%
        Une bibliothèque templatée de manipulation d'entiers à précision
        infinis.
      }
    }

\end{twenty}

%-------------------------------------------------------------------------------
% OTHER INFORMATION
%-------------------------------------------------------------------------------


\section{\enfr{Other Information}{Autres Informations}}

\subsection{\enfr{Hobbies}{Hobbys}}

\begin{twentyshort}

  \twentyitemshort{}
    {%
      Origami - Kirigami,
      Rubik's Cube,
      \enfr{Archery}{Tir à l'arc},
      \enfr{%
        Self-hosting various services on a VPS using NixOS.
      }{%
        Hébergement de services divers sur un VPS à l'aide de NixOS.
      },
    }

\end{twentyshort}

\subsection{\enfr%
  {Achievements \& Extra-curriculars}
  {Activités Extracurriculaires}
}

\begin{twenty}

  \twentyitem{\jan-\jul'20}
    {\enfr{Chief Tiger Maintainer}{Chef Mainteneur Tiger}}
    {EPITA}
    {%
      \enfr{%
        A small team of assistants is chosen each year to improve the project,
        keep it up-to-date, as well as presenting each stage of the project to
        the students and guide them along the way.
      }{%
        Une petite équipe d'assistants est choisie chaque année pour améliorer
        le projet, le mettre à jour, ainsi que de présenter chaque étape du
        sujet aux étudiants ainsi que d'assurer leur encadrement.
      }
    }

  \twentyitem{2012}
    {\enfr{Maths Olympiads (School Level)}{Olympiades de Maths (Collège)}}
    {Rouen, France}
    {%
      \enfr{%
        First place at the \textit{René Merckhoffer} contest of Normandy.

        This lead to a couple of two-weeks formation with \textit{Animath} on
        Olympic-maths.
      }{%
        Première place au concours \textit{René Merckhoffer} de Normandie.

        M'a amené à réaliser deux formations de deux semaines avec
        \textit{Animath} sur les maths olympiques.
      }
    }

\end{twenty}

%-------------------------------------------------------------------------------
% SECOND PAGE EXAMPLE
%-------------------------------------------------------------------------------

%\newpage % Start a new page

%\makeprofile % Print the sidebar

%\section{Other information}

%\subsection{Review}

% Lorem ipsum.

%\section{Other information}

%\subsection{Review}

% Lorem ipsum.

%-------------------------------------------------------------------------------

\end{document}
